\documentclass{article}

\title{chapter 1: 1.8.2 Programming assignments}
\author{Nan Ziqian
    \and 3210104676}
\date{\today}


\begin{document}
\maketitle
\section{B}

    after running ``test\_bisection.cpp''

    \begin{verbatim}
        f1:
        root: 0.860334
        width of interval: 7.13406e-13
        number of iteration: 40
        f(root): 3.84937e-12

        f2:
        root: 0.641186
        width of interval: 1.81717e-12
        number of iteration: 38
        f(root): -1.47882e-13

        f3:
        root: 1.82938
        width of interval: 3.63798e-12
        number of iteration: 38
        f(root): 2.41585e-13
        
        f4:
        root: 4
        width of interval: 8.8221e-13
        number of iteration: 41
    \end{verbatim}

    It can be seen that the 4th function does not have a root on [0,4].

\section{C}

    after running ``test\_Newton.cpp''

    \begin{verbatim}
        root near 4.5: 
        root: 4.49341
        f(root): 8.88178e-16
        number of iteration: 4

        root near 7.7:
        root: 4.49341
        f(root): 8.88178e-16
        number of iteration: 4
    \end{verbatim}

\section{D}
    \begin{verbatim}
        f1:
        root_n: 3.14159
        root_n_1: 3.14159
        f(root): -5.25802e-13
        number of iteration: 29

        f2:
        root_n: 1.30633
        root_n_1: 1.30633
        f(root): -6.52367e-12
        number of iteration: 11

        f3:
        root_n: -0.188685
        root_n_1: -0.188685
        f(root): 2.08722e-14
        number of iteration: 8
    \end{verbatim}

    You can find new root if you change f1's initial points, this is because $\sin(x/2)-1$ is periodic by $4\pi$.

    If you set the initial points bigger, Secant Method won't find any root. But if you set them smaller, you can find a root every $\pi$.
    This is because $e^x$ is sufficiently small to affect root and $\tan(x)$ is periodic by $\pi$.

    If you set the second initial point $x_1$ to 0.5, you can find the other root around 0.455.

\section{E}
    \begin{verbatim}
        bisection:
        root: 0.166166
        width of interval: 1.81899e-12
        number of iteration: 38
        f(root): -3.64153e-13

        newton method:
        root: 0.166166
        f(root): 2.34479e-13
        number of iteration: 8

        secant method:
        root_n: 0.166166
        root_n_1: 0.166166
        f(root): 0
        number of iteration: 5

        f(c-0.005) = 0.0986519
        f(c+0.005) = -0.0985668
    \end{verbatim}

    All three method find the root 0.166166, and you can find that $f(c-0.005)*f(c+0.005)<0$,
    therefore the root must be in $[c-0.005,c+0.005]$ and $|c-root|<0.01$.

\section{F}
\subsection{(a)}
    \begin{verbatim}
        root: 0.575473
        f(root): 9.59233e-13
        number of iteration: 2
    \end{verbatim}
\subsection{(b)}
    \begin{verbatim}
        root: 0.578907
        f(root): 0
        number of iteration: 3
    \end{verbatim}

\subsection{(c)}
    \begin{verbatim}
        root_n: 0.578907
        root_n_1: 0.578907
        f(root): -6.98464e-12
        number of iteration: 6

        root_n: 2.56269
        root_n_1: 2.56269
        f(root): 1.42109e-14
    \end{verbatim}

    As you can see, if you set both the initials around 2.5, the secant method will find the root 2.56269.
    But if you keep the $33^\circ$ as one initial, you can only find 0.578907, this is because $33^\circ$ is already very close to the root 0.578907,
    and you can't change the primary variable too much.

\end{document}
